\documentclass[oneside, final, 12pt]{article}
\usepackage{geometry}
\geometry{a4paper, top=2cm, right= 1.5cm, bottom=2cm, left=2cm}
\usepackage[T2A]{fontenc}
\usepackage[russian]{babel}
\usepackage{indentfirst}
\usepackage{array}

\newcommand\Section[1]{
\refstepcounter{section}
\section*{\raggedright
\arabic{section} #1}
\addcontentsline{toc}{section}{%
\arabic{section} #1}
}
\renewcommand{\labelenumi}{\arabic{section}.\arabic{subsection}.\arabic{enumi}}
\renewcommand{\labelenumii}{\asbuk{enumii})}
\renewcommand{\labelenumiii}{\arabic{enumiii})}


\begin{document}
\begin{center}
    \section*{{ЧАСТНОЕ ТЕХНИЧЕСКОЕ ЗАДАНИЕ \\НА ИНТЕГРАЦИЮ МОДУЛЯ «СОЗДАНИЯ РАССЫЛОК И НАРЯДОВ»}}
\end{center}
\newpage
\tableofcontents
\newpage
\section*{ОСНОВНЫЕ ПОНЯТИЯ И ОПРЕДЕЛЕНИЯ}
\noindent
\begin{tabular}{|l|c|p{0.8\linewidth}|} 
 \hline
 АС УДИ & -- & Автоматизированная система управления данными об изделии. АС УДИ обеспечивает создание (разработку), получение, безопасное хранение, преобразование. сопровождение конструкторских. технологических. производственных, эксплуатационных и других данных об изделии и их предоставление потребителям в соответствии с установленными правилами информации
[ГОСТ Р 58300-2018, Термины и определения, п. 2]
 \\
 \hline
 КД & -- & Конструкторский документ, который в отдельности или в совокупности с другими документами определяет конструкцию изделия и имеет содержательную и реквизитную части, в том числе установленные подписи [ГОСТ 2.001-2013, статья 3.1.2] \\
 \hline
 Наряд & -- & Планово-учетный документ, определяющий параметры размножения и рассылки конструкторской документации \\ 
 \hline
 Объект & -- & Идентифицированная (именованная) совокупность данных в информационной системе, обладающая набором атрибутов (характеристик) и предполагающая определенный метод обработки [ГОСТ 2.053-2013, статья 3.1.4] \\ 
 \hline
 ПМ & -- & Программный модуль \\ 
 \hline
 ПО & -- & Программное обеспечение, совокупность программ системы обработки информации и программных документов, необходимых для их эксплуатации \\ 
 \hline
 Рассылка & -- & Документ, задающий правила формирования наряда, в части количества высылаемых экземпляров КД по адресам \\ 
 \hline
 СУИД & -- & Система управления инженерными данными \\ 
 \hline
\end{tabular}

\newpage
\Section{ОБЩИЕ СВЕДЕНИЯ}
\subsection{Полное наименование Системы и ее условное обозначение}
Полное наименование Системы: Модуль создания рассылок и нарядов.

Краткое наименование Системы: ПМ «Наряды и рассылки».
\subsection{Наименование организации Заказчика и Разработчика}
Заказчик – Акционерное общество «Санкт-Петербургское морское бюро машиностроения «Малахит». Адрес Заказчика: 196135, г. Санкт-Петербург, ул. Фрунзе, д. 18.

Исполнитель – Общество с ограниченной ответственностью «Ирисофт». Адрес исполнителя: 197376, ул. Профессора Попова д. 23, лит. М.
\subsection{Перечень документов, на основании которых создается Система}
Указ Президента Российской Федерации от 30.03.2022 № 166 «О мерах по обеспечению технологической независимости и безопасности критической информационной инфраструктуры Российской Федерации».

Техническое задание на создание автоматизированной системы управления данными об изделии на платформе отечественного инженерного ПО.
\subsection{Плановые сроки начала и окончания работ по созданию АС}
Дата начала Работ:

Плановая дата окончания Работ:


\newpage
\Section{ЦЕЛИ И НАЗНАЧЕНИЕ СОЗДАНИЯ АВТОМАТИЗИРОВАННОЙ СИСТЕМЫ}
\subsection{Цели создания ПМ «Наряды и рассылки»}
\begin{enumerate}
    \item Основными целями разработки ПМ «Наряды и рассылки» являются:
    \begin{enumerate}
        \item автоматизация процесса выпуска планово-учетных документов наряд на размножение и отправку документации в среде АС УДИ;
        \item использование функционала СУИД по извлечению информации об отправке КД по адресам;
        \item сокращение времени создания нарядов за счет автоматического их заполнения КД из СУИД.
    \end{enumerate}
\end{enumerate}
\subsection{Назначение ПМ «Наряды и рассылки»}
\begin{enumerate}
    \item ПМ «Наряды и рассылки» предназначен для создания, редактирования и сохранения нарядов на размножение и отправку документации
\end{enumerate}

\newpage
\Section{ХАРАКТЕРИСТИКА ОБЪЕКТОВ АВТОМАТИЗАЦИИ}
Разработка ПМ «Наряды и рассылки» является составной частью работ по созданию автоматизированной системы управления данными об изделии на платформе отечественного инженерного ПО согласно техническому заданию на указанные работы.

\newpage
\Section{ТРЕБОВАНИЯ К ПМ «Наряды и рассылки»}
\subsection{Требования к функциям (задачам), выполняемым ПМ «Наряды и рассылки»}
\begin{enumerate}
    \item ПМ «Наряды и рассылки» должен обеспечивать:
    \begin{enumerate}
        \item автоматизированное формирование объекта «рассылки», задающего количество высылаемых экземпляров документов и извещений по адресам;
        \item автоматизированное формирование объекта «наряды», создаваемые по правилам, заданным в выбранном объекте «рассылка» и содержащим набор документов и извещений, высылаемых по адресам с указанием количества и площади их листов;
        \item ввод реквизитной части объекта «наряды» при их создании;
        \item сохранение объектов «рассылки» и объектов наряды в АС УДИ;
        \item отображение объекта «наряд» в виде заполненного бланка для печати в формате pdf-файла.
    \end{enumerate}
\end{enumerate}


\subsection{Требования к видам обеспечения ПМ «Наряды и рассылки»}
\subsubsection{Требования к программному обеспечению}
\begin{enumerate}
\renewcommand{\labelenumi}{\arabic{section}.\arabic{subsection}.\arabic{subsubsection}.\arabic{enumi}}
    \item В части формирования объекта «рассылка» ПМ «Наряды и рассылки» должен обеспечивать выполнение следующих требований:
    \begin{enumerate}
        \item должен быть обеспечен к команде вызова интерфейса ПМ «Наряды и рассылки» и рассылки для исполнителей, включенных в группу доступа, определенную в СУИД;
        \item должна быть обеспечена связь каждого объекта «рассылка» с конкретным объектом «изделие»;
        \item право на создание и редактирование объекта «рассылка» должно обеспечиваться правами на работу с этим объектом с СУИД;
        \item в интерфейсе ПМ «Наряды и рассылки» должно быть запрещено редактирование объекта «рассылка» после создания первого наряда на ее основе;
        \item в случае задания объекту «рассылка» состояния «выпущен» он не должен отображаться в интерфейсе ПМ «Наряды и рассылки»;
        \item ПМ «Наряды и рассылки» должен иметь графический интерфейс с отображением в виде древовидной структуры контекста (изделие) и входящих в него рассылок. При выборе конкретной рассылки должны отображаться заданные в ней правила отправки документации по адресам;
        \item ПМ «Наряды и рассылки» в части формирования объекта «рассылка» должен обеспечивать:
        \begin{enumerate}
            \item возможность задания для каждого объекта «рассылка» от одного до восьми адресов отправки. Адрес должен задаваться путем прямого ввода строки с использованием цифр, символов русского и английского алфавита, а также стандартных символов клавиатуры;
            \item возможность задания для каждого объекта «рассылка» одного или несколько диапазонов шестизначных конструктивных групп с заданным начальным значением (от) и конечным значением (до);
            \item возможность задания для каждого диапазона конструктивных групп количества высылаемых открытых документов (имеющих гриф «несекретно») и количества высылаемых закрытых документов (имеющих грифы «секретно», «сов. секретно» или пометку ДСП) по каждому из заданных адресов;
            \item возможность задания отдельной группы «извещения» с указанием для нее количества высылаемых открытых и закрытых документов по каждому из заданных адресов;
        \end{enumerate}
        \item объект «рассылка» должен храниться в СУИД. 
    \end{enumerate}
\end{enumerate}
\begin{enumerate}
\stepcounter{enumi}
\renewcommand{\labelenumi}{\arabic{section}.\arabic{subsection}.\arabic{subsubsection}.\arabic{enumi}}
    \item В части формирования объекта «наряд» ПМ «Наряды и рассылки» должен обеспечивать выполнение следующих требований:
    \begin{enumerate}
        \item должен быть обеспечен доступ к команде вызова интерфейса ПМ «Наряды и рассылки» для исполнителей, включенных в группу доступа, определенную в СУИД;
        \item должна быть обеспечена связь каждого объекта «наряд» с конкретным объектом «изделие»;
        \item право на создание и редактирование объекта «наряд» должно обеспечиваться правами на работу с этим объектом в СУИД;
        \item ПМ «Наряды и рассылки» должен иметь графический интерфейс с отображением в виде древовидной структуры контекста и входящих в него рассылок и входящих в рассылки нарядов;
        \item при выборе узла рассылки должны отображаться заданные в ней правила отправки документации по адресам;
        \item при выборе узла наряда должно отображаться его содержимое;
        \item ПМ «Наряды и рассылки» в части формирования объекта «наряд» должен обеспечивать возможность создания объекта «наряд» на основе одного выбранного объекта «рассылка» в интерфейсе ПМ «Наряды и рассылки». При создании наряда должен быть введен плановый заказ (с необязательным указанием этапа проектирования по этому плановому заказу или символа «*» – все этапы);
        \item должны быть обеспечены следующие виды поиска:
        \begin{enumerate}
            \item основной поиск должен осуществляться по обозначению и наименованию объектов с учетом следующих правил:
            \begin{itemize}
                \item должно быть соответствие плановых заказов искомых объектов атрибуту «плановый заказ», указанному при создании объекта «наряд»;
                \item в список результатов должны выводиться только последние версии объектов;
                \item объекты должны иметь состоянии «в разнарядке», т.е. состояние документов, прошедших все проверки.
            \end{itemize}
            \item должна быть возможность включить дополнительный критерий поиска по дате искомых документов;
            \item должна быть обеспечена возможность отключить один или несколько основных критериев:
            \begin{enumerate}
                \item отключения критерия последней версии -- в список результатов поиска выводятся все версии объектов;
                \item отключение критерия планового заказа –- поиск без учета совпадения плановых заказов;
                \item отключение критерия состояния «в разнарядке» -- поиск объектов в состоянии «выпущен», т.е. документов, находящихся в архиве.
            \end{enumerate}
            \item должна быть возможность переключения поиска между двумя типами объектов:
            \begin{enumerate}
                \item тип Извещение включает в себя и объекты типа бюллетень;
                \item тип КД включает в себя как сами объекты типа КД так все его подтипы, указанные в настройках.
            \end{enumerate}
            \item после получения списка результатов поиска должна быть обеспечена возможность выбора найденных объектов для включения их в объект «наряд». Добавленные объекты в окне ПМ «Наряды и рассылки» должны представлять собой табличные данные в виде таблицы 3.1.
            
            \item
            \begin{enumerate}
                \item 
                \item 
            \end{enumerate}
            \item 
            \begin{enumerate}
                \item 
                \item 
                \item 
                \item 
            \end{enumerate}
            \item 
            \begin{enumerate}
                \item 
                \item  
            \end{enumerate}
            \item 
            \begin{enumerate}
                \item 
                \item 
                \item 
            \end{enumerate}
            \item 
            \begin{enumerate}
                \item 
                \item 
                \item 
            \end{enumerate}
            \item 
            \item 
            \item 
            \item 
            \item 
            \item 
            \item 
        \end{enumerate}
    \end{enumerate}
\end{enumerate}
\subsection{Общие технические требования к ПМ «Наряды и рассылки»}
\subsubsection{Требования к защите информации от несанкционированного доступа}
\subsubsection{Требования к сохранности информации при авариях}

\newpage
\Section{СОСТАВ И СОДЕРЖАНИЕ РАБОТ ПО РАЗРАБОТКЕ ПМ «НАРЯДЫ И РАССЫЛКИ»}

\newpage
\Section{ПОРЯДОК РАЗРАБОТКИ ПМ «НАРЯДЫ И РАССЫЛКИ»}

\newpage
\Section{ПОРЯДОК КОНТРОЛЯ И ПРИЕМКИ АВТОМАТИЗИРОВАННОЙ СИСТЕМЫ}
\end{document}